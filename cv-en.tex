%% Template for a CV
%% Author: Rob J Hyndman

\documentclass[10pt,a4paper,]{article}
\usepackage[scaled=0.86]{DejaVuSansMono}
\usepackage[sfdefault,lf,t]{carlito}

% Change color to blue
\usepackage{color,xcolor}
\definecolor{headcolor}{HTML}{990000}

\usepackage{ifxetex,ifluatex}
\usepackage{fixltx2e} % provides \textsubscript
\ifnum 0\ifxetex 1\fi\ifluatex 1\fi=0 % if pdftex
  \usepackage[T1]{fontenc}
  \usepackage[utf8]{inputenc}
\else % if luatex or xelatex
  \ifxetex
    \usepackage{mathspec}
  \else
    \usepackage{fontspec}
  \fi
  \defaultfontfeatures{Ligatures=TeX,Scale=MatchLowercase}
\fi

\usepackage[utf8]{inputenc}
\usepackage[T1]{fontenc}

% use upquote if available, for straight quotes in verbatim environments
\IfFileExists{upquote.sty}{\usepackage{upquote}}{}
% use microtype if available
\IfFileExists{microtype.sty}{%
\usepackage[]{microtype}
\UseMicrotypeSet[protrusion]{basicmath} % disable protrusion for tt fonts
}{}
\PassOptionsToPackage{hyphens}{url} % url is loaded by hyperref
\usepackage[unicode=true,hidelinks]{hyperref}
\urlstyle{same}  % don't use monospace font for urls
\usepackage{geometry}
\geometry{left=1.75cm,right=1.75cm,top=2.2cm,bottom=2cm}

\usepackage{longtable,booktabs}
% Fix footnotes in tables (requires footnote package)
\IfFileExists{footnote.sty}{\usepackage{footnote}\makesavenoteenv{long table}}{}
\IfFileExists{parskip.sty}{%
\usepackage{parskip}
}{% else
\setlength{\parindent}{0pt}
\setlength{\parskip}{6pt plus 2pt minus 1pt}
}
\setlength{\emergencystretch}{3em}  % prevent overfull lines
\providecommand{\tightlist}{%
  \setlength{\itemsep}{0pt}\setlength{\parskip}{0pt}}
\setcounter{secnumdepth}{0}

% set default figure placement to htbp
\makeatletter
\def\fps@figure{htbp}
\makeatother



\date{febrero 2022}

\DeclareUnicodeCharacter{0301}{\'{e}}

\usepackage{paralist,ragged2e,datetime}
\usepackage[hyphens]{url}
\usepackage{fancyhdr,enumitem,pifont}
\usepackage[compact,small,sf,bf]{titlesec}

\RaggedRight
\sloppy

% Header and footer
\pagestyle{fancy}
\makeatletter
\lhead{\sf\textcolor[gray]{0.4}{Curriculum Vitae: \@name}}
\rhead{\sf\textcolor[gray]{0.4}{\thepage}}
\cfoot{}
\def\headrule{{\color[gray]{0.4}\hrule\@height\headrulewidth\@width\headwidth \vskip-\headrulewidth}}
\makeatother

% Header box
\usepackage{tabularx}

\makeatletter
\def\name#1{\def\@name{#1}}
\def\info#1{\def\@info{#1}}
\makeatother
\newcommand{\shadebox}[3][.9]{\fcolorbox[gray]{0}{#1}{\parbox{#2}{#3}}}

\usepackage{calc}
\newlength{\headerboxwidth}
\setlength{\headerboxwidth}{\textwidth}
%\addtolength{\headerboxwidth}{0.2cm}
\makeatletter
\def\maketitle{
\thispagestyle{plain}
\vspace*{-1.4cm}
\shadebox[0.9]{\headerboxwidth}{\sf\color{headcolor}\hfil
\hbox to 0.98\textwidth{\begin{tabular}{l}
\\[-0.3cm]
\LARGE\textbf{\@name}
\\[0.1cm]\large Professor \& Researcher\\[0.6cm]
\normalsize\textbf{Curriculum Vitae}\\
\normalsize febrero 2022
\end{tabular}
\hfill\hbox{\fontsize{9}{12}\sf
\begin{tabular}{@{}rl@{}}
\@info
\end{tabular}}}\hfil
}
\vspace*{0.2cm}}
\makeatother

% Section headings
\titlelabel{}
\titlespacing{\section}{0pt}{1.5ex}{0.5ex}
\titleformat*{\section}{\color{headcolor}\large\sf\bfseries}
\titleformat*{\subsection}{\color{headcolor}\sf\bfseries}
\titlespacing{\subsection}{0pt}{1ex}{0.5ex}

% Miscellaneous dimensions
\setlength{\parskip}{0ex}
\setlength{\parindent}{0em}
\setlength{\headheight}{15pt}
\setlength{\tabcolsep}{0.15cm}
\clubpenalty = 10000
\widowpenalty = 10000
\setlist{itemsep=1pt}
\setdescription{labelwidth=1.2cm,leftmargin=1.5cm,labelindent=1.5cm,font=\rm}

% Make nicer bullets
\renewcommand{\labelitemi}{\ding{228}}

\usepackage{booktabs,fontawesome5}
%\usepackage[t1,scale=0.86]{sourcecodepro}

\name{José Ramón Martínez Batlle}
\def\imagetop#1{\vtop{\null\hbox{#1}}}
\info{%
\raisebox{-0.05cm}{\imagetop{\faIcon{map-marker-alt}}} &  \imagetop{\begin{tabular}{@{}l@{}}School
of Geography, Autonomous University of Santo Domingo\end{tabular}}\\ %
\faIcon{home} & \href{http://geografiafisica.org}{geografiafisica.org}\\% %
\faIcon{phone-alt} & +1809 535 8273 ext 4281\\%
\faIcon{envelope} & \href{mailto:joseramon@geografiafisica.org}{\nolinkurl{joseramon@geografiafisica.org}}\\%
\faIcon{twitter} & \href{https://twitter.com/geografiard}{@geografiard}\\%
\faIcon{github} & \href{https://github.com/geofis}{geofis}\\%
%
%
%
}


%\usepackage{inconsolata}


\setlength\LTleft{0pt}
\setlength\LTright{0pt}

% Pandoc CSL macros
\newlength{\cslhangindent}
\setlength{\cslhangindent}{1.5em}
\newlength{\csllabelwidth}
\setlength{\csllabelwidth}{3em}
\newenvironment{CSLReferences}[3] % #1 hanging-ident, #2 entry spacing
 {% don't indent paragraphs
  \setlength{\parindent}{0pt}
  % turn on hanging indent if param 1 is 1
  \ifodd #1 \everypar{\setlength{\hangindent}{\cslhangindent}}\ignorespaces\fi
  % set entry spacing
  \ifnum #2 > 0
  \setlength{\parskip}{#2\baselineskip}
  \fi
 }%
 {}
\usepackage{calc}
\newcommand{\CSLBlock}[1]{#1\hfill\break}
\newcommand{\CSLLeftMargin}[1]{\parbox[t]{\csllabelwidth}{\hfill #1~}}
\newcommand{\CSLRightInline}[1]{\parbox[t]{\linewidth - \cslhangindent - \csllabelwidth}{#1}\vspace{0.8ex}}
\newcommand{\CSLIndent}[1]{\hspace{\cslhangindent}#1}


\def\endfirstpage{\newpage}

\begin{document}
\maketitle


\hypertarget{biography}{%
\section{Biography}\label{biography}}

Professor and researcher, trained as an engineer and geographer with
advanced studies in geography.

I teach geomorphology, biogeography and general physical geography,
among other subjects, at Autonomous University of Santo Domingo (UASD).
I have also taught at national and international universities on nature
conservation and disaster risk management.

I do research in fields such as geomorphology, biogeography,
conservation and management of natural resources, disaster risk
management and environmental changes. My main motivation is to answer
research questions, produce and publish new knowledge, and develop
accessible science for my students. I have published research papers,
participated in science conferences, collaborated in writing books and
book chapters, and advised undergraduate and graduate students in these
disciplines.

For my teaching and research work, I rely on geographic information
technology, such as optical and radar remote sensing, GIS,
photogrammetry, GNSS positioning, cloud-computing platforms. I also use
spatial statistics, machine learning and deep learning techniques. Most
of my workflows are supported by R and Python analytical scripts. In
general, I only use free and/or open source software under Linux
environment. Recently, I am developing my own software packages and
creating DIY-hardware.

\hypertarget{employments}{%
\section{Employments}\label{employments}}

\begin{longtable}{@{\extracolsep{\fill}}ll}
2004 - Present & \parbox[t]{0.85\textwidth}{%
\textbf{Profesor / Investigador}\hfill{\footnotesize Universidad Autónoma de Santo Domingo}\newline
  Santo Domingo\par%
  \empty%
\vspace{\parsep}}\\
2008 - 2013 & \parbox[t]{0.85\textwidth}{%
\textbf{Project Manager for International Cooperation}\hfill{\footnotesize Agencia Española de Cooperación Internacional para el Desarrollo}\newline
  Santo Domingo\par%
  \empty%
\vspace{\parsep}}\\
2006 - 2008 & \parbox[t]{0.85\textwidth}{%
\textbf{Natural Resources Management Technician}\hfill{\footnotesize Fondo Mixto Hispano-Panameño de Cooperación Técnica}\newline
  Panamá\par%
  \empty%
\vspace{\parsep}}\\
2005 - 2006 & \parbox[t]{0.85\textwidth}{%
\textbf{Department manager}\hfill{\footnotesize Oficina Nacional de Planificación. Secretariado Técnico de la Presidencia (STP)}\newline
  Santo Domingo\par%
  \empty%
\vspace{\parsep}}\\
1996 - 1999 & \parbox[t]{0.85\textwidth}{%
\textbf{Department manager and Project Leader}\hfill{\footnotesize Dirección Nacional de Parques}\newline
  Santo Domingo\par%
  \empty%
\vspace{\parsep}}\\
\end{longtable}

\hypertarget{education}{%
\section{Education}\label{education}}

\begin{longtable}{@{\extracolsep{\fill}}ll}
1999 - 2012 & \parbox[t]{0.85\textwidth}{%
\textbf{Ph.D in Physical Geography}\hfill{\footnotesize Universidad de Sevilla}\newline
  Sevilla\par%
  \empty%
\vspace{\parsep}}\\
2001 - 2003 & \parbox[t]{0.85\textwidth}{%
\textbf{Degree in Geography}\hfill{\footnotesize Universidad de Sevilla}\newline
  Sevilla\par%
  \empty%
\vspace{\parsep}}\\
1997 - 1999 & \parbox[t]{0.85\textwidth}{%
\textbf{Master in Nature Conservation and Management}\hfill{\footnotesize Universidad Internacional de Andalucía}\newline
  Huelva\par%
  \empty%
\vspace{\parsep}}\\
1990 - 1994 & \parbox[t]{0.85\textwidth}{%
\textbf{Electrical Engineer}\hfill{\footnotesize Instituto Tecnológico de Santo Domingo}\newline
  Santo Domingo\par%
  \empty%
\vspace{\parsep}}\\
\end{longtable}

\hypertarget{works}{%
\section{Works}\label{works}}

\hypertarget{bibliography}{}
\leavevmode\hypertarget{ref-Jose_Ramon_Martinez-Batlle_108106153}{}%
\CSLLeftMargin{1. }
\CSLRightInline{Martínez-Batlle, J. R. (2021). Cartografía
geomorfológica de detalle de un río tropical usando fotografías aéreas
de resolución centimétrica y deep learning. In \emph{XX Jornada de
Investigación Científica. Universidad Autónoma de Santo Domingo
(UASD)}.}

\leavevmode\hypertarget{ref-Jose_Ramon_Martinez-Batlle_108106177}{}%
\CSLLeftMargin{2. }
\CSLRightInline{Martínez-Batlle, J. R. (2021). Deforestación y fuego en
República Dominicana durante el siglo XXI. In \emph{XX Jornada de
Investigación Científica. Universidad Autónoma de Santo Domingo
(UASD)}.}

\leavevmode\hypertarget{ref-Jose_Ramon_Martinez-Batlle_108106129}{}%
\CSLLeftMargin{3. }
\CSLRightInline{Martínez-Batlle, J. R. (2021). Deformación superficial
en la ciudad de Santo Domingo usando small baseline subset (SBAS). In
\emph{XX Jornada de Investigación Científica. Universidad Autónoma de
Santo Domingo (UASD)}.}

\leavevmode\hypertarget{ref-Jose_Ramon_Martinez-Batlle_108106194}{}%
\CSLLeftMargin{4. }
\CSLRightInline{Martínez-Batlle, J. R. (2021). Revisitando RTK para
todos. Sistema base-rover GNSS-RTK multibanda y multiconstelación de
bajo costo. In \emph{XX Jornada de Investigación Científica. Universidad
Autónoma de Santo Domingo (UASD)}.}

\leavevmode\hypertarget{ref-https:ux2fux2fdoi.orgux2f10.5281ux2fzenodo.5694017}{}%
\CSLLeftMargin{5. }
\CSLRightInline{Martínez-Batlle, J. R. (2021).
\emph{geofis/forest-loss-fire-reproducible: First release}. Zenodo.
\url{https://doi.org/10.5281/ZENODO.5694017}}

\leavevmode\hypertarget{ref-https:ux2fux2fdoi.orgux2f10.5281ux2fzenodo.5682104}{}%
\CSLLeftMargin{6. }
\CSLRightInline{Martínez-Batlle, J. R. (2021). \emph{Dataset for: Forest
loss and fire in the Dominican Republic during the 21st Century}.
Zenodo. \url{https://doi.org/10.5281/ZENODO.5682104}}

\leavevmode\hypertarget{ref-Jose_Ramon_Martinez-Batlle_108096224}{}%
\CSLLeftMargin{7. }
\CSLRightInline{Martínez-Batlle, J. R. (2021). Fuego y deforestación.
Vistazo nacional, acercamiento a Valle Nuevo. In \emph{Grupo de
discusión e intercambio científico {``Plagiodontia.''} Museo Nacional de
Historia Natural {``Prof.~Eugenio de Jesús Marcano''}}.}

\leavevmode\hypertarget{ref-Batlle_2021}{}%
\CSLLeftMargin{8. }
\CSLRightInline{Martínez-Batlle, J. R. (2021). \emph{Forest loss and
fire in the Dominican Republic during the 21st Century}.
\url{https://doi.org/10.1101/2021.06.15.448604}}

\leavevmode\hypertarget{ref-Mart_nez_Batlle_2021}{}%
\CSLLeftMargin{9. }
\CSLRightInline{Martínez-Batlle, J. R., \& van-der-Hoek, Y. (2021).
Plant community associations with morpho-topographic, geological and
land use attributes in a semi-deciduous tropical forest of the Dominican
Republic. \emph{Neotropical Biodiversity}, \emph{7}(1), 465--475.
\url{https://doi.org/10.1080/23766808.2021.1987769}}

\leavevmode\hypertarget{ref-Martinez_Batlle_2020}{}%
\CSLLeftMargin{10. }
\CSLRightInline{Martínez-Batlle, J. R., \& van-der-Hoek, Y. (2020).
\emph{Plant community associations with environmental variables in a
semi-deciduous tropical forest of the Dominican Republic}.
\url{https://doi.org/10.1101/2020.08.04.235390}}

\leavevmode\hypertarget{ref-C_mara_Artigas_2020}{}%
\CSLLeftMargin{11. }
\CSLRightInline{Cámara-Artigas, R., Díaz-del-Olmo, F., \&
Martínez-Batlle, J. R. (2020). TBRs, a methodology for the multi-scalar
cartographic analysis of the distribution of plant formations.
\emph{Boletín de La Asociación de Geógrafos Españoles}, \emph{85}.
\url{https://doi.org/10.21138/bage.2915}}

\leavevmode\hypertarget{ref-Jose_Ramon_Martinez-Batlle_108106017}{}%
\CSLLeftMargin{12. }
\CSLRightInline{Martínez-Batlle, J. R. (2019). Estimación de la
granulometría de carga gruesa superficial mediante fotografías de alta
resolución tomadas por UAV. In \emph{XVIII Jornada de Investigación
Científica. Universidad Autónoma de Santo Domingo (UASD)}.}

\leavevmode\hypertarget{ref-Jose_Ramon_Martinez-Batlle_108105769}{}%
\CSLLeftMargin{13. }
\CSLRightInline{Martínez-Batlle, J. R. (2019). Geomorfología de detalle
de un tramo de 1 kilómetro del río Mana, proximidades de Villa
Altagracia, República Dominicana. In \emph{XVIII Jornada de
Investigación Científica. Universidad Autónoma de Santo Domingo
(UASD)}.}

\leavevmode\hypertarget{ref-Jose_Ramon_Martinez-Batlle_108106039}{}%
\CSLLeftMargin{14. }
\CSLRightInline{Martínez-Batlle, J. R. (2019). Historia de un rechazo:
ordenación de comunidades plantas de Ocoa, revisitada. In \emph{XVIII
Jornada de Investigación Científica. Universidad Autónoma de Santo
Domingo (UASD)}.}

\leavevmode\hypertarget{ref-Jose_Ramon_Martinez-Batlle_108106059}{}%
\CSLLeftMargin{15. }
\CSLRightInline{Martínez-Batlle, J. R. (2019). RTK para todos. Unidad
GNSS de bajo costo basado en Raspberry Pi: primeros resultados y
potenciales aplicaciones. In \emph{XVIII Jornada de Investigación
Científica. Universidad Autónoma de Santo Domingo (UASD)}.}

\leavevmode\hypertarget{ref-Jose_Ramon_Martinez-Batlle_76604230}{}%
\CSLLeftMargin{16. }
\CSLRightInline{Martínez-Batlle, J. R. (2019). Drainage Rearrangement as
a Driver of Geomorphological Evolution During the Upper Pleistocene in a
Small Tropical Basin. \emph{Journal of Geography and Geology}.
\url{https://doi.org/10.31223/OSF.IO/PFZVQ}}

\leavevmode\hypertarget{ref-Jose_Ramon_Martinez-Batlle_108096060}{}%
\CSLLeftMargin{17. }
\CSLRightInline{Martínez-Batlle, J. R. (2019). Reorganización del
drenaje durante el Pleistoceno Superior en la cuenca del río Ocoa. In
\emph{2do. Congreso Internacional de Geología Aplicada de República
Dominicana. Sociedad Dominicana de Geología}.}

\leavevmode\hypertarget{ref-Mart_nez_Batlle_2018}{}%
\CSLLeftMargin{18. }
\CSLRightInline{Martínez-Batlle, J. R., \& van-der-Hoek, Y. (2018).
Clusters of high abundance of plants detected from local indicators of
spatial association (LISA) in a semi-deciduous tropical forest.
\emph{PLOS ONE}, \emph{13}(12), e0208780.
\url{https://doi.org/10.1371/journal.pone.0208780}}

\leavevmode\hypertarget{ref-Jose_Ramon_Martinez-Batlle_108008287}{}%
\CSLLeftMargin{19. }
\CSLRightInline{Martínez-Batlle, J. R. (2018). Reorganización del
drenaje durante el Pleistoceno Superior en la cuenca del río Ocoa. In
\emph{XVII Jornada de Investigación Científica. Universidad Autónoma de
Santo Domingo (UASD)}.}

\leavevmode\hypertarget{ref-Jose_Ramon_Martinez-Batlle_76604201}{}%
\CSLLeftMargin{20. }
\CSLRightInline{Martínez-Batlle, J. R. (2018). Digital photogrammetry of
historical aerial photographs using open-source software. In
\emph{EarthArXiv}. \url{https://doi.org/10.31223/osf.io/bna95}}

\leavevmode\hypertarget{ref-Jose_Ramon_Martinez-Batlle_108164639}{}%
\CSLLeftMargin{21. }
\CSLRightInline{Martínez-Batlle, J. R. (2018). Análisis con escasos
recursos y técnicas libres: diversidad y asociación de comunidades de
plantas con variables ambientales. In \emph{XIV Congreso Internacional
de Investigación Científica. Ministerio de Educación Superior, Ciencia y
Tecnología (MESCyT)}.}

\leavevmode\hypertarget{ref-Jose_Ramon_Martinez-Batlle_108164664}{}%
\CSLLeftMargin{22. }
\CSLRightInline{Martínez-Batlle, J. R., \& van-der-Hoek, Y. (2018).
Asociación de comunidades de plantas con variables ambientales en un
bosque semideciduo tropical de República Dominicana. In \emph{XIV
Congreso Internacional de Investigación Científica. Ministerio de
Educación Superior, Ciencia y Tecnología (MESCyT)}.}

\leavevmode\hypertarget{ref-Jose_Ramon_Martinez-Batlle_108164842}{}%
\CSLLeftMargin{23. }
\CSLRightInline{Martínez-Batlle, J. R., Navarro-Morales, S. Q.,
Lubertazzi, D., Sánchez-Rosario, A., Sosa-Hernández, D. A.,
Manzueta-Acevedo, P. K., Martínez-Uribe, J. I., \& García-Disla, S. O.
(2018). Diversidad de hormigas en el Jardín Botánico Nacional y el
Parque Mirador del Norte. In \emph{XIV Congreso Internacional de
Investigación Científica. Ministerio de Educación Superior, Ciencia y
Tecnología (MESCyT)}.}

\leavevmode\hypertarget{ref-Jose_Ramon_Martinez-Batlle_108164731}{}%
\CSLLeftMargin{24. }
\CSLRightInline{Martínez-Batlle, J. R. (2018). Software de código
abierto y gratuito para fotogrametría digital y SIG en la producción de
cartografía geomorfológica, utilizando como fuentes fotografías áereas
tradicionales adquiridas con cámara fotogramétrica. Aplicación a la
cuenca del río Ocoa. In \emph{XIV Congreso Internacional de
Investigación Científica. Ministerio de Educación Superior, Ciencia y
Tecnología (MESCyT)}.}

\leavevmode\hypertarget{ref-rodriguez2018analisis}{}%
\CSLLeftMargin{25. }
\CSLRightInline{Rodríguez, E. B., Martínez-Batlle, J. R.,
Cámara-Artigas, R., \& Silva-Pérez, R. (2018). Análisis estadístico
espacial de las pérdidas en la ocupación de las formaciones adehesadas
en Sierra Morena (1956-2007). \emph{Bosque Mediterráneo y Humedales:
Paisaje, Evolución y Conservación: Aportaciones Desde La Biogeografía},
711--721.}

\leavevmode\hypertarget{ref-del2018manglares}{}%
\CSLLeftMargin{26. }
\CSLRightInline{Díaz-del-Olmo, F., Cámara-Artigas, R., Martínez-Batlle,
J. R., \& Morón-Monge, M. del C. (2018). Manglares de Chiriquí (costa
del Pacífico, Panamá): Diagnóstico biogeomorfológico aplicado a la
conservación de costas tropicales. \emph{América Latina En Las últimas
décadas: Procesos y Retos}, 81--100.}

\leavevmode\hypertarget{ref-Jose_Ramon_Martinez-Batlle_108105108}{}%
\CSLLeftMargin{27. }
\CSLRightInline{Martínez-Batlle, J. R., \& van-der-Hoek, Y. (2017).
Asociaciones de especies con variables ambientales en un bosque
semideciduo tropical de República Dominicana. In \emph{XVI Jornada de
Investigación Científica. Universidad Autónoma de Santo Domingo
(UASD)}.}

\leavevmode\hypertarget{ref-Jose_Ramon_Martinez-Batlle_108105172}{}%
\CSLLeftMargin{28. }
\CSLRightInline{Martínez-Batlle, J. R., \& van-der-Hoek, Y. (2017).
Clusters de alta abundancia de plantas detectados mediante indicadores
locales de asociación espacial (LISA) en un bosque semideciduo tropical.
In \emph{XVI Jornada de Investigación Científica. Universidad Autónoma
de Santo Domingo (UASD)}.}

\leavevmode\hypertarget{ref-Jose_Ramon_Martinez-Batlle_108520061}{}%
\CSLLeftMargin{29. }
\CSLRightInline{Martínez-Batlle, J. R. (2017). \emph{Introducción a la
fotogrametría aplicada a imágenes tomadas desde sistemas de aeronaves
pilotadas a distancia (RPAS) o drones}.}

\leavevmode\hypertarget{ref-Jose_Ramon_Martinez-Batlle_108008472}{}%
\CSLLeftMargin{30. }
\CSLRightInline{Martínez-Batlle, J. R., \& van-der-Hoek, Y. (2017).
Detección de hotspots de biodiversidad dentro de transectos mediante
autocorrelación espacial, y su relación con el hábito de plantas
arborescentes. In \emph{XIII Congreso Internacional de Investigación
Científica. Ministerio de Educación Superior, Ciencia y Tecnología
(MESCyT)}.}

\leavevmode\hypertarget{ref-Jose_Ramon_Martinez-Batlle_108008490}{}%
\CSLLeftMargin{31. }
\CSLRightInline{Martínez-Batlle, J. R. (2017). Algoritmo desarrollado en
R para generar imágenes de satélite libres de nubes. In \emph{XIII
Congreso Internacional de Investigación Científica. Ministerio de
Educación Superior, Ciencia y Tecnología (MESCyT)}.}

\leavevmode\hypertarget{ref-Jose_Ramon_Martinez-Batlle_108008425}{}%
\CSLLeftMargin{32. }
\CSLRightInline{Martínez-Batlle, J. R. (2017). Diseño de muestreo con
caracterización semiautomática de variables del territorio mediante R y
QGIS. Aplicación en áreas urbanas. In \emph{XIII Congreso Internacional
de Investigación Científica. Ministerio de Educación Superior, Ciencia y
Tecnología (MESCyT)}.}

\leavevmode\hypertarget{ref-Jose_Ramon_Martinez-Batlle_108010451}{}%
\CSLLeftMargin{33. }
\CSLRightInline{Martínez-Batlle, J. R. (2017). Análisis de diversidad de
plantas leñosas y arborescentes de la cuenca media del río Ocoa. In
\emph{IX Congreso de la Biodiversidad Caribeña. Universidad Autónoma de
Santo Domingo (UASD)}.}

\leavevmode\hypertarget{ref-Jose_Ramon_Martinez-Batlle_108010431}{}%
\CSLLeftMargin{34. }
\CSLRightInline{Heredia, Y., \& Martínez-Batlle, J. R. (2017).
Biodiversidad y caracterización florística en etapas serales de la
vegetación del parque nacional Mirador Norte. In \emph{IX Congreso de la
Biodiversidad Caribeña. Universidad Autónoma de Santo Domingo (UASD)}.}

\leavevmode\hypertarget{ref-Jose_Ramon_Martinez-Batlle_108104972}{}%
\CSLLeftMargin{35. }
\CSLRightInline{Martínez-Batlle, J. R. (2016). Estimación del porcentaje
de cobertura arbórea de la cuenca del río Ocoa mediante clasificación de
imágenes Landsat y Pléiades. In \emph{XV Jornada de Investigación
Científica. Universidad Autónoma de Santo Domingo (UASD)}.}

\leavevmode\hypertarget{ref-Jose_Ramon_Martinez-Batlle_108095974}{}%
\CSLLeftMargin{36. }
\CSLRightInline{Martínez-Batlle, J. R. (2016). Valle Nuevo es \ldots{}
cambios. In \emph{Panel {``Valle Nuevo es agua.''} Academia de Ciencias
de la República Dominicana}.}

\leavevmode\hypertarget{ref-Jose_Ramon_Martinez-Batlle_108041435}{}%
\CSLLeftMargin{37. }
\CSLRightInline{Martínez-Batlle, J. R., Rojas-Valerio, W., \&
De-Aza-Concepción, M. (2016). Clasificación de microcuencas de orden 2 a
partir de sus parámetros morfométricos. Relación con factores
litológicos. In \emph{XII Congreso Internacional de Investigación
Científica. Ministerio de Educación Superior, Ciencia y Tecnología
(MESCyT)}.}

\leavevmode\hypertarget{ref-Jose_Ramon_Martinez-Batlle_108041097}{}%
\CSLLeftMargin{38. }
\CSLRightInline{Martínez-Batlle, J. R., \& Laurencio-Girón, G. (2016).
Desprendimiento El Canal, cuenca del río Ocoa: análisis del cambio en un
movimiento de ladera mediante fotogrametría digital y geoestadística. In
\emph{XII Congreso Internacional de Investigación Científica. Ministerio
de Educación Superior, Ciencia y Tecnología (MESCyT)}.}

\leavevmode\hypertarget{ref-Jose_Ramon_Martinez-Batlle_108041325}{}%
\CSLLeftMargin{39. }
\CSLRightInline{Martínez-Batlle, J. R. (2016). Determinación
semiautomática de la superficie quemada mediante teledetección de imagen
Landsat OLI/TIRS: el incendio de Valle Nuevo de mayo de 2015. In
\emph{XII Congreso Internacional de Investigación Científica. Ministerio
de Educación Superior, Ciencia y Tecnología (MESCyT)}.}

\leavevmode\hypertarget{ref-Jose_Ramon_Martinez-Batlle_108041498}{}%
\CSLLeftMargin{40. }
\CSLRightInline{Martínez-Batlle, J. R. (2016). Diversidad y patrones
espaciales de vegetación arborescente en bosque semideciduo: análisis
con estadística univariada y geoestadística. In \emph{XII Congreso
Internacional de Investigación Científica. Ministerio de Educación
Superior, Ciencia y Tecnología (MESCyT)}.}

\leavevmode\hypertarget{ref-Jose_Ramon_Martinez-Batlle_108010543}{}%
\CSLLeftMargin{41. }
\CSLRightInline{Martínez-Batlle, J. R., \& Laurencio-Girón, G. (2016).
Estimación del porcentaje de cobertura arbórea de la cuenca del río Ocoa
mediante clasificación de imágenes Landsat y Pléiades. In \emph{XII
Congreso Internacional de Investigación Científica. Ministerio de
Educación Superior, Ciencia y Tecnología (MESCyT)}.}

\leavevmode\hypertarget{ref-Jose_Ramon_Martinez-Batlle_108010605}{}%
\CSLLeftMargin{42. }
\CSLRightInline{Martínez-Batlle, J. R. (2016). Terraza del Pleistoceno
Superior del tramo medio de la cuenca del río Ocoa: resultados
preliminares de dataciones por radiocarbono utilizando AMS. In \emph{XII
Congreso Internacional de Investigación Científica. Ministerio de
Educación Superior, Ciencia y Tecnología (MESCyT)}.}

\leavevmode\hypertarget{ref-Jose_Ramon_Martinez-Batlle_76609746}{}%
\CSLLeftMargin{43. }
\CSLRightInline{Martínez-Batlle, J. R., Santos-Grullón, I.,
Laurencio-Girón, G., Cámara-Artigas, R., Valerio-Rojas, W.,
De-Aza-Concepción, M., \& Medina-Castillo, E. (2016). Diversidad alpha y
factores físicos de bosques tropicales con contraste estacional, cuenca
del río Ocoa. \emph{Anuario de Investigaciones Científicas. Universidad
Autónoma de Santo Domingo (UASD)}.}

\leavevmode\hypertarget{ref-Jose_Ramon_Martinez-Batlle_108041630}{}%
\CSLLeftMargin{44. }
\CSLRightInline{Martínez-Batlle, J. R. (2015). Biodiversidad,
geoestadística y factores litogeomorfológicos. Una aplicación a
muestreos de flora en República Dominicana. In \emph{XIV Jornadas de
Investigación Científica. Universidad Autónoma de Santo Domingo
(UASD)}.}

\leavevmode\hypertarget{ref-Jose_Ramon_Martinez-Batlle_108095931}{}%
\CSLLeftMargin{45. }
\CSLRightInline{Martínez-Batlle, J. R. (2015). Bosques en régimen
tropical y con contraste estacional de la cuenca del río Ocoa.
Diversidad alpha y factores explicativos. In \emph{Grupo de discusión e
intercambio científico {``Plagiodontia.''} Museo Nacional de Historia
Natural . Eugenio de Jesús Marcano''}.}

\leavevmode\hypertarget{ref-Jose_Ramon_Martinez-Batlle_108046725}{}%
\CSLLeftMargin{46. }
\CSLRightInline{Barrios-Rodríguez, M., \& Martínez-Batlle, J. R. (2015).
Análisis de componentes principales de deslizamientos superficiales en
laderas de la cuenca del río Ocoa, República Dominicana. In \emph{XI
Congreso Internacional de Investigación Científica. Ministerio de
Educación Superior, Ciencia y Tecnología (MESCyT)}.}

\leavevmode\hypertarget{ref-Jose_Ramon_Martinez-Batlle_108047433}{}%
\CSLLeftMargin{47. }
\CSLRightInline{Martínez-Batlle, J. R., Rojas-Valerio, W., \&
Medina-Castillo, E. (2015). Análisis de los componentes principales de
depósitos aluviales de la cuenca del río Ocoa, República Dominicana. In
\emph{XI Congreso Internacional de Investigación Científica. Ministerio
de Educación Superior, Ciencia y Tecnología (MESCyT)}.}

\leavevmode\hypertarget{ref-Jose_Ramon_Martinez-Batlle_108047709}{}%
\CSLLeftMargin{48. }
\CSLRightInline{Martínez-Batlle, J. R., Rojas-Valerio, W., \&
Medina-Castillo, E. (2015). Avances en el conocimiento de la dinámica
aluvial del río Ocoa a partir de análisis estadístico de la tipología y
morfometría de 10,000 gravas y bloques. In \emph{XI Congreso
Internacional de Investigación Científica. Ministerio de Educación
Superior, Ciencia y Tecnología (MESCyT)}.}

\leavevmode\hypertarget{ref-Jose_Ramon_Martinez-Batlle_108052894}{}%
\CSLLeftMargin{49. }
\CSLRightInline{Martínez-Batlle, J. R., Santos-Grullón, I.,
Laurencio-Girón, G., Cámara-Artigas, R., De-Aza-Concepción, M.,
Rojas-Valerio, W., \& Medina-Castillo, E. (2015). Bosques en régimen
tropical y con contraste estacional de la cuenca del río Ocoa (I).
Diversidad alpha de plantas leñosas, palmas y Cactaceae y factores
explicativos, a partir del análisis de 24 muestreos de campo. In
\emph{XI Congreso Internacional de Investigación Científica. Ministerio
de Educación Superior, Ciencia y Tecnología (MESCyT)}.}

\leavevmode\hypertarget{ref-Jose_Ramon_Martinez-Batlle_108052927}{}%
\CSLLeftMargin{50. }
\CSLRightInline{Martínez-Batlle, J. R. (2015). El Catálogo de plantas
con semillas de las Indias Occidentales (Acevedo- Rodríguez y Strong,
2012) {``pasado por R.''} In \emph{XI Congreso Internacional de
Investigación Científica. Ministerio de Educación Superior, Ciencia y
Tecnología (MESCyT)}.}

\leavevmode\hypertarget{ref-Jose_Ramon_Martinez-Batlle_108049160}{}%
\CSLLeftMargin{51. }
\CSLRightInline{Martínez-Batlle, J. R., \& Laurencio-Girón, G. (2015).
La fotointerpretación digital y la estadística inferencial como
herramientas de determinación preliminar de tipos litológicos: estudio a
partir de la densidad de drenaje y la relación de bifurcación en la
ciudad de Ocoa y su entorno. In \emph{XI Congreso Internacional de
Investigación Científica. Ministerio de Educación Superior, Ciencia y
Tecnología (MESCyT)}.}

\leavevmode\hypertarget{ref-Jose_Ramon_Martinez-Batlle_108049302}{}%
\CSLLeftMargin{52. }
\CSLRightInline{Martínez-Batlle, D., José Ramón Herrera-Hernández, \&
Laurencio-Girón, G. (2015). Terrazas fluviales, cauces trenzados,
vertientes y piedemontes: avances en la delimitación detallada de
elementos morfológicos de la ciudad de Ocoa y su entorno mediante
fotointerpretación digital. In \emph{XI Congreso Internacional de
Investigación Científica. Ministerio de Educación Superior, Ciencia y
Tecnología (MESCyT)}.}

\leavevmode\hypertarget{ref-Jose_Ramon_Martinez-Batlle_108053251}{}%
\CSLLeftMargin{53. }
\CSLRightInline{Martínez-Batlle, J. R. (2015). Depende la diferenciación
de especies entre dos muestreos de la litología o el relieve? Análisis
aplicado a 24 muestreos de bosques mesófilos de la cuenca del río Ocoa.
In \emph{XI Congreso Internacional de Investigación Científica.
Ministerio de Educación Superior, Ciencia y Tecnología (MESCyT)}.}

\leavevmode\hypertarget{ref-Jose_Ramon_Martinez-Batlle_108049360}{}%
\CSLLeftMargin{54. }
\CSLRightInline{Martínez-Batlle, J. R., Santos-Grullón, I.,
Rojas-Valerio, W., Medina-Castillo, E., \& De-Aza-Concepción, M. (2015).
Encontramos una captura fluvial identificando y midiendo {``callaos''}
de río? Evidencia de una captura fluvial mediante análisis estadístico
de gravas y bloques: caso de los arroyos Parra y Naranjal. In \emph{XI
Congreso Internacional de Investigación Científica. Ministerio de
Educación Superior, Ciencia y Tecnología (MESCyT)}.}

\leavevmode\hypertarget{ref-Jose_Ramon_Martinez-Batlle_108050134}{}%
\CSLLeftMargin{55. }
\CSLRightInline{Martínez-Batlle, J. R., Laurencio-Girón, G., \&
Medina-Castillo, E. (2015). Está cambiando la hidráulica del río Ocoa?
Evidencia geomorfológica mediante el análisis estadístico de barras
aluviales de la ciudad de Ocoa y Los Pilones. In \emph{XI Congreso
Internacional de Investigación Científica. Ministerio de Educación
Superior, Ciencia y Tecnología (MESCyT)}.}

\leavevmode\hypertarget{ref-Jose_Ramon_Martinez-Batlle_108053568}{}%
\CSLLeftMargin{56. }
\CSLRightInline{Martínez-Batlle, J. R., \& Laurencio-Girón, G. (2015).
Pueden contribuir la fotointerpretación digital y la estadística
inferencial en la detección temprana de un deslizamiento? El caso de la
Vuelta de la Paloma, cuenca del río Ocoa. In \emph{XI Congreso
Internacional de Investigación Científica. Ministerio de Educación
Superior, Ciencia y Tecnología (MESCyT)}.}

\leavevmode\hypertarget{ref-Jose_Ramon_Martinez-Batlle_108053371}{}%
\CSLLeftMargin{57. }
\CSLRightInline{Martínez-Batlle, J. R., Santos-Grullón, I.,
Laurencio-Girón, G., \& De-Aza-Concepción, M. (2015). Qué nos dice la
geoestadística sobre la estructura, la biodiversidad y los factores del
bosque? Análisis aplicado a 24 muestreos de bosques mesófilos de la
cuenca del río Ocoa. In \emph{XI Congreso Internacional de Investigación
Científica. Ministerio de Educación Superior, Ciencia y Tecnología
(MESCyT)}.}

\leavevmode\hypertarget{ref-Jose_Ramon_Martinez-Batlle_108046657}{}%
\CSLLeftMargin{58. }
\CSLRightInline{Medina-Castillo, E., \& Martínez-Batlle, J. R. (2015).
Qué tanto aumenta la información geomorfológica la fotogrametría
digital? Geomorfología de la ciudad de Barahona mediante
fotointerpretación digital: avances en la correcta clasificación y la
precisión geométrica. In \emph{XI Congreso Internacional de
Investigación Científica. Ministerio de Educación Superior, Ciencia y
Tecnología (MESCyT)}.}

\leavevmode\hypertarget{ref-Jose_Ramon_Martinez-Batlle_108050214}{}%
\CSLLeftMargin{59. }
\CSLRightInline{Martínez-Batlle, J. R. (2015). Se asocian las
características y los factores de la vegetación con los modelos
rango-abundancia de bosques de plantas leñosas, palmas y Cactaceae?
Análisis aplicado a 24 muestreos de bosques mesófilos de la cuenca del
río Ocoa. In \emph{XI Congreso Internacional de Investigación
Científica. Ministerio de Educación Superior, Ciencia y Tecnología
(MESCyT)}.}

\leavevmode\hypertarget{ref-Jose_Ramon_Martinez-Batlle_108053606}{}%
\CSLLeftMargin{60. }
\CSLRightInline{Martínez-Batlle, J. R., \& Laurencio-Girón, G. (2015).
El deslizamiento de la Vuelta de la Paloma, cuenca del río Ocoa,
República Dominicana: aplicación de la fotointerpretación digital y
técnicas estadísticas para su detección preliminar. In \emph{XV
Encuentro de Geógrafos de América Latina}.}

\leavevmode\hypertarget{ref-Jose_Ramon_Martinez-Batlle_108100939}{}%
\CSLLeftMargin{61. }
\CSLRightInline{Martínez-Batlle, J. R., \& Laurencio-Girón, G. (2014).
Comparación de la relación de bifurcación y densidad de drenaje en
cuencas vertientes del entorno de Ocoa (República Dominicana) mediante
fotointerpretación digital: aplicaciones en estudios geomorfológicos y
geológicos. In \emph{XIII Jornadas de Investigación Científica.
Universidad Autónoma de Santo Domingo (UASD)}.}

\leavevmode\hypertarget{ref-Jose_Ramon_Martinez-Batlle_108100916}{}%
\CSLLeftMargin{62. }
\CSLRightInline{Santos-Grullón, I., \& Martínez-Batlle, J. R. (2014).
Análisis comparado de las redes de drenaje de la cuenca del río Ocoa
(República Dominicana), generadas a partir del mapa topográfico nacional
1:50,000 y el modelo digital de elevaciones ASTER (30 m), usando ArcGIS
y GRASS GIS. In \emph{XIII Jornadas de Investigación Científica.
Universidad Autónoma de Santo Domingo (UASD)}.}

\leavevmode\hypertarget{ref-Jose_Ramon_Martinez-Batlle_108100865}{}%
\CSLLeftMargin{63. }
\CSLRightInline{Martínez-Batlle, J. R., Santos-Grullón, I.,
Laurencio-Girón, G., \& Cámara-Artigas, R. (2014). Bosques en régimen
tropical y con contraste estacional de la cuenca del río Ocoa (I).
Resultados preliminares de la diversidad alpha aplicado a 24 muestreos
de campo. In \emph{XIII Jornadas de Investigación Científica.
Universidad Autónoma de Santo Domingo (UASD)}.}

\leavevmode\hypertarget{ref-Jose_Ramon_Martinez-Batlle_108101000}{}%
\CSLLeftMargin{64. }
\CSLRightInline{Medina-Castillo, E., \& Martínez-Batlle, J. R. (2014).
Geomorfología de la ciudad de Barahona (República Dominicana) mediante
fotointerpretación digital: avances en la correcta clasificación y la
precisión geométrica. In \emph{XIII Jornadas de Investigación
Científica. Universidad Autónoma de Santo Domingo (UASD)}.}

\leavevmode\hypertarget{ref-Jose_Ramon_Martinez-Batlle_108100990}{}%
\CSLLeftMargin{65. }
\CSLRightInline{Martínez-Batlle, J. R., Herrera-Hernández, D., \&
Laurencio-Girón, G. (2014). Geomorfología del pueblo de Ocoa y su
entorno: cartografía de detalle mediante reconocimiento de terreno y
fotointerpretación digital. In \emph{XIII Jornadas de Investigación
Científica. Universidad Autónoma de Santo Domingo (UASD)}.}

\leavevmode\hypertarget{ref-Jose_Ramon_Martinez-Batlle_108101655}{}%
\CSLLeftMargin{66. }
\CSLRightInline{Santos-Grullón, I., \& Martínez-Batlle, J. R. (2014).
Comparación de las redes de drenaje obtenidas del mapa topográfico
nacional y un MDE-ASTER de resolución media, cuenca del río Ocoa,
República Dominicana: análisis morfométrico, estadística descriptiva y
geoestadística. In \emph{X Congreso Internacional de Investigación
Científica. Ministerio de Educación Superior, Ciencia y Tecnología
(MESCyT)}.}

\leavevmode\hypertarget{ref-Jose_Ramon_Martinez-Batlle_108101322}{}%
\CSLLeftMargin{67. }
\CSLRightInline{Martínez-Batlle, J. R., \& Aranguren-Cotallo, A. (2014).
El Servidor de Mapas del Instituto Geográfico Universitario (SIGU)
adquirido mediante proyecto FONDOCyT. In \emph{X Congreso Internacional
de Investigación Científica. Ministerio de Educación Superior, Ciencia y
Tecnología (MESCyT)}.}

\leavevmode\hypertarget{ref-Jose_Ramon_Martinez-Batlle_108101266}{}%
\CSLLeftMargin{68. }
\CSLRightInline{Martínez-Batlle, J. R., \& Herrera-Hernández, D. (2014).
Homogeneización, análisis de tendencia y superficies continuas de
precipitación y temperatura de la cuenca del río Ocoa, usando técnicas
de estadística inferencial y geoestadística. In \emph{X Congreso
Internacional de Investigación Científica. Ministerio de Educación
Superior, Ciencia y Tecnología (MESCyT)}.}

\leavevmode\hypertarget{ref-Jose_Ramon_Martinez-Batlle_108101335}{}%
\CSLLeftMargin{69. }
\CSLRightInline{Laurencio-Girón, G., \& Martínez-Batlle, J. R. (2014).
Ortorrectificación de fotogramas aéreos de la cuenca del río Ocoa
donados por el INDRHI. Aplicaciones en geomorfología y biogeografía. In
\emph{X Congreso Internacional de Investigación Científica. Ministerio
de Educación Superior, Ciencia y Tecnología (MESCyT)}.}

\leavevmode\hypertarget{ref-Jose_Ramon_Martinez-Batlle_108101339}{}%
\CSLLeftMargin{70. }
\CSLRightInline{Medina-Castillo, E., \& Martínez-Batlle, J. R. (2014).
Ortorrectificación de fotogramas aéreos de la sierra de Bahoruco donados
por el INDRHI. Aplicaciones en geomorfología y biogeografía. In \emph{X
Congreso Internacional de Investigación Científica. Ministerio de
Educación Superior, Ciencia y Tecnología (MESCyT)}.}

\leavevmode\hypertarget{ref-Jose_Ramon_Martinez-Batlle_108102009}{}%
\CSLLeftMargin{71. }
\CSLRightInline{Cámara-Artigas, R., Martínez-Batlle, J. R., \&
Herrera-Hernández, D. (2014). Análisis cuantitativo de la diversidad de
fanerófitas y caméfitas en ocho transectos de la Hoya de Enriquillo,
República Dominicana: rango-abundancia, análisis comparado y
agrupamiento. In \emph{X Congreso Internacional de Investigación
Científica. Ministerio de Educación Superior, Ciencia y Tecnología
(MESCyT)}.}

\leavevmode\hypertarget{ref-Jose_Ramon_Martinez-Batlle_108102026}{}%
\CSLLeftMargin{72. }
\CSLRightInline{Martínez-Batlle, J. R., Herrera-Hernández, D.,
Cámara-Artigas, R., Santos-Grullón, I., Laurencio-Girón, G., \&
Medina-Castillo, E. (2014). Estudio geobotánico comparado de las lomas
la Cruz y de Cholo, cuenca del río Ocoa, República Dominicana: bosques
meso-tropófilos en morfosistema poligenético de karst y flysch. In
\emph{VIII Congreso de Biodiversidad Caribeña. Universidad Autónoma de
Santo Domingo (UASD)}.}

\leavevmode\hypertarget{ref-Jose_Ramon_Martinez-Batlle_108102032}{}%
\CSLLeftMargin{73. }
\CSLRightInline{Martínez-Batlle, J. R., Cámara-Artigas, R.,
Santos-Grullón, I., Herrera-Hernández, D., Medina-Castillo, E., \&
Laurencio-Girón, G. (2014). Estudio geobotánico de Monteada Nueva,
Cortico y Cachote, sierra de Bahoruco, República Dominicana: superficies
corrosivas, pedimentos y bosques nublados amenazados en karst de montaña
tropical. In \emph{VIII Congreso de Biodiversidad Caribeña. Universidad
Autónoma de Santo Domingo (UASD)}.}

\leavevmode\hypertarget{ref-Jose_Ramon_Martinez-Batlle_76606520}{}%
\CSLLeftMargin{74. }
\CSLRightInline{Quílez-Caballero, Á., Martínez-Batlle, J. R., \&
Cámara-Artigas. (2014). Caracterización biogeográfica y distribución de
los bosques nublados de montaña en Bahoruco Oriental, República
Dominicana. In \emph{VIII Congreso Español de Biogeografía. Biogeografía
de Sistemas Litorales. Dinámica y Conservación. Vicerrectorado de
Investigación de la Universidad de Sevilla. Grupo de Geografía Física de
la Asociación Española de Geografía (AGE)}.}

\leavevmode\hypertarget{ref-Jose_Ramon_Martinez-Batlle_76609683}{}%
\CSLLeftMargin{75. }
\CSLRightInline{Martínez-Batlle, J. R., Cámara-Artigas, R.,
Santos-Grullón, I., Laurencio-Girón, G., Herrera-Hernández, D.,
Medina-Castillo, E., \& Laurencio-Girón, G. (2014). Geomorfología y
geobotánica de Monteada Nueva, Cortico y Cachote, sierra de Bahoruco,
República Dominicana: superficies corrosivas, pedimentos y bosques
nublados amenazados en karst de montaña media tropical. \emph{Anuario de
Investigaciones Científicas. Universidad Autónoma de Santo Domingo
(UASD)}.}

\leavevmode\hypertarget{ref-Jose_Ramon_Martinez-Batlle_108102314}{}%
\CSLLeftMargin{76. }
\CSLRightInline{Gómez-Ponce, C., Cámara-Artigas, R., Martínez-Batlle, J.
R., \& Díaz-del-Olmo, F. (2014). Metodología para el estudio de sistemas
de arrecifes de coral con imágenes de satélite LandSat: sistema
arrecifal de Cabedelo-Cabo Branco (Joao Pessoa, estado de Pernambuco,
Brasil). In \emph{VIII Congreso Español de Biogeografía. Biogeografía de
Sistemas Litorales. Dinámica y Conservación. Vicerrectorado de
Investigación de la Universidad de Sevilla. Grupo de Geografía Física de
la Asociación Española de Geografía (AGE)}.}

\leavevmode\hypertarget{ref-Jose_Ramon_Martinez-Batlle_108102047}{}%
\CSLLeftMargin{77. }
\CSLRightInline{Martínez-Batlle, J. R., Cámara-Artigas, R.,
Santos-Grullón, I., Herrera-Hernández, D., Medina-Castillo, E., \&
Laurencio-Girón, G. (2013). Geomorfología y geobotánica de Monteada
Nueva, Cortico y Cachote, sierra de Bahoruco, República Dominicana:
superficies corrosivas, pedimentos y bosques nublados amenazados en
karst de montaña media tropical. In \emph{XIII Jornadas de Investigación
Científica. Universidad Autónoma de Santo Domingo (UASD)}.}

\leavevmode\hypertarget{ref-Jose_Ramon_Martinez-Batlle_108102200}{}%
\CSLLeftMargin{78. }
\CSLRightInline{Martínez-Batlle, J. R., \& Cámara-Artigas, R. (2012).
Estudio comparativo de las formaciones vegetales de montaña media
tropical sobre relieves calizos karstificados. Sierra de Bahoruco
(República Dominicana)-Sierra Madre Oriental (Tamaulipas, México). In
\emph{VII Congreso español de Biogeografía, Pirineo 2012. Las zonas de
montaña: gestión y biodiversidad. Grup de Recerca en Àrees de Muntanya i
Paisatge, Departament de Geografía, Universitat Autònoma de Barcelona.
Fundació Catalunya Caixa. MónNatura Pirineus}.}

\leavevmode\hypertarget{ref-batlle2012sierra}{}%
\CSLLeftMargin{79. }
\CSLRightInline{Martínez-Batlle, J. R. (2012). \emph{Sierra de Bahoruco
Occidental, República Dominicana: estudio biogeomorfológico y estado de
conservación de su parque nacional} {[}Tesis doctoral{]}. Universidad de
Sevilla.}

\leavevmode\hypertarget{ref-ponce2010metodologia}{}%
\CSLLeftMargin{80. }
\CSLRightInline{Gómez-Ponce, C., Martínez-Batlle, J. R., \&
Cámara-Artigas, R. (2010). Metodología para la realización de
batimetrías con imágenes de satélite Landsat: Ejemplos de sistemas
arrecifales coralinos al este de República Dominicana. \emph{Mapping},
\emph{142}, 46--50.}

\leavevmode\hypertarget{ref-del2009diagnostico}{}%
\CSLLeftMargin{81. }
\CSLRightInline{Díaz-del-Olmo, F., Cámara-Artigas, R., \&
Martínez-Batlle, J. R. (2009). Diagnóstico y ordenación del Manglar de
la Provincia de Chiriquí (Panamá). \emph{Geografía: Ciencia de La Tierra
Para La Sostenibilidad}, 55--78.}

\leavevmode\hypertarget{ref-diaz2009evenement}{}%
\CSLLeftMargin{82. }
\CSLRightInline{Díaz-del-Olmo, F., Martínez-Batlle, J. R.,
Cámara-Artigas, R., \& Salomon, J.-N. (2009). L'évènement climatique
extrême tropical (ECET) du 21 novembre 2006: analyse
hydro-géomorphologique et gestion intégrée des bassins versants
hydrographiques tropicaux. \emph{Les Cahiers d'Outre-Mer}, \emph{2},
219--240.}

\leavevmode\hypertarget{ref-Jose_Ramon_Martinez-Batlle_107991988}{}%
\CSLLeftMargin{83. }
\CSLRightInline{Martínez-Batlle, J. R., Díaz-del-Olmo, F.,
Cámara-Artigas, R., González, L., \& Ramos, M. del C. (2008).
Directrices de ordenación para la gestión integrada de las cuencas de
los ríos Indio y Miguel de la Borda. In \emph{Fondo Mixto
Hispano-Panameño de Cooperación, Autoridad Nacional del Ambiente
(ANAM)}.}

\leavevmode\hypertarget{ref-Jose_Ramon_Martinez-Batlle_107991418}{}%
\CSLLeftMargin{84. }
\CSLRightInline{Cámara-Artigas, R., Bejarano-Palma, R., Martínez-Batlle,
J. R., \& Díaz-del-Olmo, F. (2006). Estructura y geobotánica de la
vegetación de bosques tropófilos y helófilos tropicales en antiguos
humedales colmatados: laguna de Limón y En medio (Hoya de Enriquillo,
República Dominicana). In \emph{IV Congreso Español de Biogeografía.
Avances en biogeografía. Ávila}.}

\leavevmode\hypertarget{ref-Jose_Ramon_Martinez-Batlle_108102758}{}%
\CSLLeftMargin{85. }
\CSLRightInline{Martínez-Batlle, J. R. (2005). Formaciones vegetales
relictas: pinares y bosques nublados entre 1.000 y 2.400 m. en la sierra
de Bahoruco (suroeste de República Dominicana). In \emph{I Congreso
Internacional de Investigación Científica. Ministerio de Educación
Superior, Ciencia y Tecnología (MESCyT)}.}

\leavevmode\hypertarget{ref-Jose_Ramon_Martinez-Batlle_108102743}{}%
\CSLLeftMargin{86. }
\CSLRightInline{Martínez-Batlle, J. R. (2005). Jimaní: claves
geomorfológicas de la riada. In \emph{I Congreso Internacional de
Investigación Científica. Ministerio de Educación Superior, Ciencia y
Tecnología (MESCyT)}.}

\leavevmode\hypertarget{ref-Jose_Ramon_Martinez-Batlle_108102772}{}%
\CSLLeftMargin{87. }
\CSLRightInline{Martínez-Batlle, J. R. (2005). Cambios decenales
(1973-99) de formaciones vegetales tropicales en la provincia Pedernales
(República Dominicana). aplicación de técnicas cartográficas y sensores
remotos (Landsat MSS, ETM+). In \emph{V Congreso de Biodiversidad
Caribeña. Universidad Autónoma de Santo Domingo (UASD)}.}

\leavevmode\hypertarget{ref-Jose_Ramon_Martinez-Batlle_108102787}{}%
\CSLLeftMargin{88. }
\CSLRightInline{Martínez-Batlle, J. R. (2005). Evolución histórica de
los manglares del humedal de las bahías de Monte Cristi y Manzanillo
(parque nacional de Montecristi, República Dominicana). In \emph{V
Congreso de Biodiversidad Caribeña. Universidad Autónoma de Santo
Domingo (UASD)}.}

\leavevmode\hypertarget{ref-Jose_Ramon_Martinez-Batlle_108102793}{}%
\CSLLeftMargin{89. }
\CSLRightInline{Martínez-Batlle, J. R. (2005). Nuevas tendencias en la
ordenación de recursos naturales y áreas protegidas en el caribe
(Panamá, Costa Rica y República Dominicana). In \emph{V Congreso de
Biodiversidad Caribeña. Universidad Autónoma de Santo Domingo (UASD)}.}

\leavevmode\hypertarget{ref-artigas2005desarrollo}{}%
\CSLLeftMargin{90. }
\CSLRightInline{Cámara-Artigas, R., Martínez-Batlle, J. R., \&
Díaz-del-Olmo, F. (2005). \emph{Desarrollo sostenible y medio ambiente
en República Dominicana: medios naturales, manejo histórico,
conservación y protección}. Consejo Superior de Investigaciones
Cientı́ficas, Escuela de Estudios Hispano-Americanos, Universidad de
Sevilla. \url{https://books.google.com.do/books?id=tfe4CCivLzYC}}

\leavevmode\hypertarget{ref-2005Mdcr}{}%
\CSLLeftMargin{91. }
\CSLRightInline{Díaz-del-Olmo, F., Cámara-Artigas, R., Martínez-Batlle,
J. R., \& Gómez-Ponce, C. (2005). \emph{Mapa de comunidades, recursos
turísticos y servicios del Golfo de Kuna Yala {[}Material
cartográfico{]}}. Fondo mixto hispano-panameño de cooperación.}

\leavevmode\hypertarget{ref-2005Mdod}{}%
\CSLLeftMargin{92. }
\CSLRightInline{Díaz-del-Olmo, F., Cámara-Artigas, R., Martínez-Batlle,
J. R., \& Gómez-Ponce, C. (2005). \emph{Mapa de ordenación de los
recursos naturales del Golfo de Kuna Yala {[}Material cartográfico{]} =
Environmental management mapping, Kuna Yala Gulf (Panama)}. Fondo mixto
hispano-panameño de cooperación.}

\leavevmode\hypertarget{ref-Jose_Ramon_Martinez-Batlle_108102924}{}%
\CSLLeftMargin{93. }
\CSLRightInline{Cámara-Artigas, R., Martínez-Batlle, J. R., \&
Díaz-del-Olmo, F. (2004). Comportamiento dinámico estacional del humedal
tropical de la laguna de Cabral (Hoya de Enriquillo, República
Dominicana): sector San Cristóbal. In \emph{III Congreso Español de
Biogeografía. Comunicaciones. Isla de Txatxarramendi (Sukarrieta),
Reserva de la Biosfera de Urdaibai. Universidad del País Vasco.}}

\leavevmode\hypertarget{ref-batlle2004arrecifes}{}%
\CSLLeftMargin{94. }
\CSLRightInline{Gómez-Ponce, C., Martínez-Batlle, J. R., Cámara-Artigas,
R., \& Díaz-del-Olmo, F. (2004). Arrecifes frangeantes de Paso de
Catuano (Parque Nacional del Este, República Dominicana).
\emph{Fronteras En Movimiento}, 307--316.}

\leavevmode\hypertarget{ref-Jose_Ramon_Martinez-Batlle_108001058}{}%
\CSLLeftMargin{95. }
\CSLRightInline{Díaz-del-Olmo, F., Cámara-Artigas, R., Martínez-Batlle,
J. R., \& Morón-Monge, M. del C. (2004). \emph{Directrices de gestioń
para la conservacioń y desarrollo integral de un humedal centroamericano
: Golfo de Montijo (litoral del Pacif́ico, Panama)́}. Agencia Española De
Cooperación Internacional.}

\leavevmode\hypertarget{ref-2004Dyrp}{}%
\CSLLeftMargin{96. }
\CSLRightInline{Díaz-del-Olmo, F., Cámara-Artigas, R., Martínez-Batlle,
J. R., \& Morón-Monge, M. del C. (2004). \emph{Directrices y
recomendaciones para el uso y gestión sostenible de los manglares de
Chiriquí (República de Panamá)}. Ministerio de Economía y Finanzas de
Panamá, Autoridad Nacional del Ambiente de Panamá, Embajada de España en
Panamá, Cooperación española.}

\leavevmode\hypertarget{ref-Jose_Ramon_Martinez-Batlle_108002351}{}%
\CSLLeftMargin{97. }
\CSLRightInline{Díaz-del-Olmo, F., Cámara-Artigas, R., \&
Martínez-Batlle, J. R. (2004). \emph{Plan de Ordenación de los Recursos
Naturales de la Provincia Pedernales}. Oficina Nacional de Planificación
(ONAPLAN) del Secretariado Técnico de la Presidencia de República
Dominicana; Agencia Española de Cooperación Internacional.}

\leavevmode\hypertarget{ref-martinez2003arrecifes}{}%
\CSLLeftMargin{98. }
\CSLRightInline{Martínez-Batlle, J. R., Gómez-Ponce, C., Cámara-Artigas,
R., \& Díaz-del-Olmo, F. (2003). Arrecifes costeros sumergidos en paso
de Catuano (Parque Nacional del Este, República Dominicana):
caracterización biosedimentaria y aplicación a la ordenación de unidades
ambientales. \emph{XI Reunión Nacional de Cuaternario}, 27--32.}

\leavevmode\hypertarget{ref-Jose_Ramon_Martinez-Batlle_108103180}{}%
\CSLLeftMargin{99. }
\CSLRightInline{Martínez-Batlle, J. R. (2003). Dinámica aluvial y
riesgos naturales por inundaciones en regiones tropicales: conos de
desbordamiento de Vicente Noble y Tamayo, República Dominicana. In
\emph{XI Reunión Nacional de Cuaternario (XI-RENAQ). Asociación Española
de Cuaternario (AEQUA)}.}

\leavevmode\hypertarget{ref-Jose_Ramon_Martinez-Batlle_108103244}{}%
\CSLLeftMargin{100. }
\CSLRightInline{Martínez-Batlle, J. R., Cámara-Artigas, R.,
Díaz-del-Olmo, F., \& Gómez-Ponce, C. (2002). Arrecifes frangeantes de
Paso de Catuano (Parque Nacional del Este, República Dominicana). In
\emph{Fronteras en movimiento. IX Coloquio Ibérico de Geografía.
Huelva}.}

\leavevmode\hypertarget{ref-del2002hatos}{}%
\CSLLeftMargin{101. }
\CSLRightInline{Díaz-del-Olmo, F., Cámara-Artigas, R., \&
Martínez-Batlle, J. R. (2002). Hatos caribeños y dehesas andaluzas.
Paisaje y estructura parcelaria. In \emph{Cuba y Andalucı́a entre las dos
orillas} (Vol. 422, p. 303). Editorial CSIC-CSIC Press.}

\leavevmode\hypertarget{ref-martinez2002sabanas}{}%
\CSLLeftMargin{102. }
\CSLRightInline{Martínez-Batlle, J. R. (2002). Sabanas de la República
Dominicana: análisis ecodinámico de patrones tipológicos y sus ecotonos
{[}Mathesis{]}. In \emph{Proyecto de investigación de Doctorado Cambios
ambientales y riesgos naturales. Departamento de geografía física y
análisis geográfico regional, Universidad de Sevilla}. Departamento de
Geografía Física y Análisis Regional, Universidad de Sevilla.}

\end{document}
